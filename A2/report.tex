\documentclass[10pt]{article}
\usepackage{graphicx}
\usepackage{amsmath}
\usepackage{amsfonts}
\usepackage{subfig}
\usepackage{listings}

\setlength{\oddsidemargin}{27mm}
\setlength{\evensidemargin}{27mm}
\setlength{\hoffset}{-1in}

\setlength{\topmargin}{27mm}
\setlength{\voffset}{-1in}
\setlength{\headheight}{0pt}
\setlength{\headsep}{0pt}

\setlength{\textheight}{235mm}
\setlength{\textwidth}{155mm}

%\pagestyle{empty}
\pagestyle{plain}

\renewcommand{\thefootnote}{\fnsymbol{footnote}}
\renewcommand{\labelitemi}{$\diamond$}

\begin{document}
\baselineskip 12pt

\begin{center}
\textbf{\Large CS 482: Computational Techniques in Biological Sequence Analysis Homework \#2}\\

\vspace{0.5cc}
{ \sc Eric Haoran Huang$^{1}$}\\

\vspace{0.2 cm}

{\small $^{1}$e48huang@uwaterloo.ca, 20880126, e48huang}
 \end{center}

\begin{abstract}
  \noindent This assignment here is meant to analyze and showcase the power of dynamic programming and the relevant sequence alignment programs.
\end{abstract}
\section*{Setup}
I used Python 3.8.10 in this run without any external dependencies besides the regular Python library dependencies.

\section*{Part 1: Edit Distance}


\end{document}
